% !TeX spellcheck = ru_RU-Russian
% !TEX program = lualatex

\documentclass[12pt]{extarticle}

\usepackage[paperheight=570mm, paperwidth=200mm, margin=3mm, showframe]{geometry}
\setlength\parindent{0pt}

\usepackage{fontspec}
\setmainfont[Ligatures=TeX]{Gentium Plus}

\newcommand{\my}[1]{#1}

\usepackage{graphicx}
\graphicspath{{images/}}


\begin{document}
    \pagestyle{empty}
    \sloppy
%    \begin{tikzpicture}
%        \node[text width=150mm, draw=red, text justified] {
            Независимая страна с 1960 (соглашения Цюриха--Лондона). Форма правления~--- президентская республика. Президент (исполнительная власть) избирается народом каждые пять лет и управляет страной вместе с кабинетом министров, которых он сам назначает.
            C~1~мая 2004 входит в ЕC. Вступил в Еврозону 1~января 2008. Является членом ООН, не является членом НАТО. Законодательная власть осуществляется Палатой представителей (Парламентом), насчитывающей 80 мест: 56 для греко-киприотов и 24 для турко-киритов. Депутаты избираются каждые пять лет прям голосования народа. С 1963 года турко-киприоты покинули Парламент, поэтому на сегодняшний день он представлен только греко-киприотами.

            %\footnotesize
            \begin{tabular}{rl}
                Πρόεδρος & Νίκος Χριστοδουλίδης\\
                \my{President} & \my{Nikos Christodoulides}\\

                Υπουργός Εξωτερικών & Κωνσταντίνος Κόμπος\\
                \my{Minister of Foreign Affairs} & \my{Konstantinos Kompos} \\

                Υπουργός Οικονομικών & Μάκης Κεραυνός\\
                \my{Minister of Finance} & \my{Makis Keravnos} \\

                Υπουργός Εσωτερικών & Κωνσταντίνος Ιωάννου\\
                \my{Minister of Interior} & \my{Konstantinos Ioannou} \\

                Υπουργός Άμυνας & Βασίλης Πάλμας\\
                \my{Minister of Defence} & \my{Vasilis Palmas} \\

                Υπουργός Παιδείας, Αθλητισμού και Νεολαίας & Αθηνά Μιχαηλίδου\\
                \my{Minister of Education, Sport and Youth} & \my{Athina Michaelidou} \\

                Υπουργός Μεταφορών, Επικοινωνιών και Έργων & Αλέξης Βαφεάδης\\
                \my{Minister of Transport, Communications and Works} & \my{Alexis Vafeades} \\

                Υπουργός Ενέργειας, Εμπορίου και Βιομηχανίας & Μιχάλης Δαμιανού\\
                \my{Minister of Energy, Commerce and Industry} & \my{Michalis Damianos} \\

                \my{(*)} Υ Γεωργίας, Αγροτικής Ανάπτυξης και Περι-ς & Μαρία Παναγιώτου\\
                \my{M of Agriculture, Rural Development and Env-t} & \my{Maria Panayiotou} \\

                Υπουργός Εργασίας και Κοινωνικών Ασφαλίσεων & Μαρίνος Μουσιούττας\\
                \my{Minister of Labour and Social Insurance} & \my{Marinos Mousiouttas} \\

                \my{(*)} Υπουργός Δικαιοσύνης και Δημόσιας Τάξεως & Κώστας Φιτσίρης\\
                \my{Minister of Justice and Public Order} & \my{Costas Fitiris} \\

                Υπουργός Υγείας & Νεόφυτος Χαραλαμπίδης\\
                \my{Minister of Health} & \my{Neophytos Charalambides} \\
            \end{tabular}

            \begin{tabular}{rl}
                1960--1977 & Αρχιεπίσκοπος Μακάριος Γʹ \\
                1977--1988 & Σπύρος Κυπριανού \\
                1988--1993 & Γιώργος Βασιλείου \my{(умер в январе 2026)}\\
                1993--2003 & Γλαύκος Κληρίδης \\
                2003--2008 & Τάσσος Παπαδόπουλος \\
                2008--2013 & Δημήτρης Χριστόφιας \\
                2013--2023 & Νίκος Αναστασιάδης \\
                2023--\phantom{2026} & Νίκος Χριστοδουλίδης \\
            \end{tabular}


            Остров в~восточной части Средиземного моря, расположен близко к~Европе, Азии
            и~Африке. Он~является третьим по~величине островом в~Средиземном море после Сардинии и~Сицилии. Находится он к~востоку от~островов Родос и~Крит, к~югу
            от~берегов Турции (на~расстоянии 75 километров), к~западу от~берегов Сирии
            (на~расстоянии 105 километров) и~к~северу от~берегов Египта (на~расстоянии 380 километров).
            Кипр имеет площадь 9\,251 квадратных километров, максимальную длину 240 километров и~максимальную ширину 100 километров.
            36{,}2\% территории Кипра находится под турецкой оккупацией.

            \includegraphics[width=\textwidth]{geography1}
            \includegraphics[width=\textwidth]{geography2}
            \includegraphics[width=\textwidth]{monasteries}

            \begin{tabular}{rp{170mm}}
                16.08.1960 & Ημέρα της Ανεξαρτησίας της Κύπρου
                \my{День независимости Кипра}\\

                21.12.1963 & Ματωμένα Χριστούγεννα
                \my{Кровавое рождество (начало межобщинных столкновений)}\\

                04.03.1964 & Άφιξη της Ειρηνευτικής Δύναμης των Ηνωμένων Εθνών
                \my{Размещение миротворческих сил ООН (ВСООНК)}\\

                15.07.1974 & Το Πραξικόπημα της 15ης Ιουλίου
                \my{Государственный переворот (свержение Макариоса III)}\\

                20.07.1974 & Η Τουρκική Εισβολή στην Κύπρο
                \my{Начало турецкого военного вторжения (Операция Аттила)}\\

                14.08.1974 & Η δεύτερη φάση της τουρκικής εισβολής
                \my{Вторая фаза вторжения и оккупация севера острова}\\

                03.08.1977 & Θάνατος του Αρχιεπισκόπου Μακαρίου Γ'
                \my{Смерть первого президента Кипра, архиепископа Макариоса III}\\

                15.11.1983 & Η Μονομερής Ανακήρυξη του Ψευδοκράτους
                \my{Одностороннее провозглашение «независимости» оккупированных территорий}\\

                14.08.1996 & Η δολοφονία του Σολωμού Σολωμού
                \my{Убийство Соломоса Солому во время протестов в Деринии}\\

                23.04.2003 & Το πρώτο άνοιγμα των οδοφραγμάτων
                \my{Первое открытие контрольно-пропускных пунктов на границе}\\

                24.04.2004 & Το Δημοψήφισμα για το Σχέδιο Ανάν
                \my{Референдум по Плану Аннана по объединению острова}\\

                01.05.2004 & Η Ένταξη της Κύπρου στην Ευρωπαϊκή Ένωση
                \my{Вступление Республики Кипр в Европейский Союз}\\

                01.01.2008 & Η Υιοθέτηση του Ευρώ στην Κύπρο
                \my{Официальный переход Кипра на валюту евро}\\

                11.07.2011 & Η έκρηξη στο Μαρί
                \my{Взрыв на военной базе «Евангелос Флоракис» в Мари}\\

                28.12.2011 & Ανακάλυψη Φυσικού Αερίου στο κοίτασμα «Αφροδίτη»
                \my{Официальное объявление об открытии газового месторождения «Афродита»}\\
            \end{tabular}


            \begin{tabular}{rp{170mm}}
                01.01 &  Πρωτοχρονιά \my{(Новый год)}
                 Οικογενειακά γεύματα, βασιλόπιτα με φλουρί, επισκέψεις σε συγγενείς
                \my{Семейные обеды, василопита с монеткой, визиты к родственникам}\\

                06.01 &  Θεοφάνεια \my{(Богоявление)}
                 Αγιασμός των υδάτων· στις ακτές γίνεται κατάδυση για τον σταυρό
                \my{Освящение воды; на побережье — ныряние за крестом}\\

                10 μέρες &  Καρναβάλι \my{(Карнавал)}
                 Παρελάσεις, μεταμφιέσεις, χοροί, γιορτές (ιδιαίτερα στη Λεμεσό)
                \my{Парады, маскарады, танцы, празднества (особенно в Лимассоле)}\\

                Πέμπτη &  Τσικνοπέμπτη \my{(Цикнопемпти)}
                 Παραδοσιακό ψήσιμο κρέατος στα κάρβουνα πριν τη Σαρακοστή
                \my{Традиционная жарка мяса на углях перед Великим постом}\\

                Δευτέρα &  Καθαρά Δευτέρα \my{(Чистый понедельник)}
                 Πέταγμα χαρταετού, νηστίσιμα φαγητά, εξόρμηση στην ύπαιθρο
                \my{Запуск воздушных змеев, постная еда, выезд на природу}\\

                25.03 &  25η Μαρτίου \my{(День греческой независимости)}
                 Στρατιωτικές και σχολικές παρελάσεις, σημαίες της Ελλάδας και της Κύπρου
                \my{Военные и школьные парады, флаги Греции и Кипра}\\

                01.04 &  1η Απριλίου \my{(Начало борьбы EOKA)}
                 Επίσημες τελετές, καταθέσεις στεφάνων, σχολικές εκδηλώσεις
                \my{Официальные церемонии, возложение венков, школьные мероприятия}\\

                Πάσχα &  Πάσχα \my{(Пасха)}
                 Αναστάσιμη λειτουργία, οικογενειακό τραπέζι, βαмμένα αυγά, ψήσιμο κρέατος· αργία τη Δευτέρα
                \my{Ночная служба, семейное застолье, крашеные яйца, шашлыки; выходной понедельник}\\

                01.05 &  Πρωτομαγιά \my{(День труда)}
                 Πικνίκ, εξορμήσεις στη φύση, μερικές φορές συγκεντρώσεις
                \my{Пикники, выезды на природу, иногда митинги}\\

                Δευτέρα &  Κατακλυσμός \my{(Катаклизмос)}
                 Γιορτή του Αγίου Πνεύματος, πανηγύρια στις παραλιακές πόλεις, παιχνίδια με νερό
                \my{Праздник Святого Духа, ярмарки в прибрежных городах, обливания водой}\\

                15.08 &  Κοίμηση της Θεοτόκου \my{(Успение Богородицы)}
                 Θρησκευτικές λειτουργίες, επισκέψεις στα χωριά, οικογενειακοί εορτασμοί
                \my{Церковные службы, поездки в деревни, семейные праздники}\\

                01.10 &  Ημέρα Ανεξαρτησίας \my{(День независимости Кипра)}
                 Στρατιωτική παρέλαση στη Λευκωσία, επίσημες εκδηλώσεις
                \my{Военный парад в Никосии, официальные мероприятия}\\

                28.10 &  28η Οκτωβρίου \my{(День «Охи»)}
                 Παρελάσεις, μνημόσυνες τελετές, πατριωτικές εκδηλώσεις
                \my{Парады, памятные церемонии, патриотические события}\\

                25.12 &  Χριστούγεννα \my{(Рождество)}
                 Οικογενειακά δείπνα, δώρα, εκκλησιαστικές λειτουργίες
                \my{Семейные ужины, подарки, церковные службы}\\
           \end{tabular}


           \begin{description}
               \item[Официальные языки.] Греческий и~турецкий.

               \item[Религии.]
               Греко-киприоты~--- православные христиане, турко-киприоты~--- мусульмане. Также проживают армяне, марониты (прибыли из~Ливана в~8--13 веках) и~приверженцы римско-католической церкви.

               \item[Дороги.] Макс. скорость на магистрали~--- 100 км/ч, макс. кол-во алкоголя~--- 22 мг на 100 мл выдыхаемого воздуха.

               \item[Электричество.] Интерконнектор соединяет Кипр, Грецию, Израиль. Также строят LNG terminal в Василикосе.

               \item[ВВП.] Процент ВВП от сельского хозяйства~--- 2\%.

               \item[ГеСИ.] 18--40 лет~--- 4~бесплатных посещения; 47--50 лет~--- 6~бесплатных посещений; дальше~--- 15~евро за раз. Приём у~врачей-специалистов стоит 6~евро, у~рентгенологов~--- 10.
               Одна бесплатная чистка зубов в~год.

               \item[Газеты.] Ежедневные: Филеефтерос,
               Политис, Симерини и~Харавги; по воскресеньям: Кафимерини;
               англоязычная: Сyprus Mail.

               \item[Телекоммуникации.] Код: 00357 (+357); телефонные компании: Cyta (АТЕН), Epic, Primetel и~Cablenet; гос. оператор: РИК.

               \item[Национальная Гвардия.] Объединенные вооруженные силы (сухопутные, морские и~авиационные). Продолжительность службы~--- 14~месяцев; для женщин-волонтёров~--- 6~месяцев.

               %                Служат только греко-киприоты. С~2008 года военная служба обязательна для всех членов Греческого Сообщества, а не только для этнических греков.
               %				Продолжительность службы составляет 14 месяцев.
               %Все мужчины в~возрасте
               %от~16~лет и~старше, являющиеся киприотами по~происхождению, обязаны получить разрешение на~выезд из~Республики Кипр от~Министерства Обороны.

               \item[Спорт.] Футбольные команды: АЕЛ (Лимассол), АПОЭЛ (Никосия), Омония (Никосия)
               и~Анортосис (Фамагуста).

               \item[Учреждения и часы работы.]
               С пн по пт, с~7:30--8:00 до до~14:00--15:00 дня.
               Аптеки: до~13:30 по ср и сб, в вск не работают, летом ночные работают до 23:00, дежурные~--- по телефону 11892.
               Банки: 8:00--14:30. Организации:
               АНК~--- Управление электроэнергетики кипра,
               Департаменты водоснабжения Кипра,
               КЕП~--- Центр обслуживания граждан (типа МФЦ).
           \end{description}

%        };
%    \end{tikzpicture}
\end{document}