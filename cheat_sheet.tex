% !TeX spellcheck = ru_RU-Russian
% !TEX program = lualatex

%\documentclass[12pt]{extarticle}
%
%\usepackage[
%paperheight=297mm, paperwidth=50mm,
%%left=3mm, right=3mm, top=8mm, bottom=8mm, %headsep=2mm,
%showframe
%]{geometry}

\documentclass{standalone}

\usepackage{booktabs}
\usepackage{multirow}
\usepackage{tikz}

\usepackage{babel}

\usepackage{fontspec}
\setmainfont[Ligatures=TeX]{Gentium Plus}

\usepackage{emoji}
\setemojifont{Apple Color Emoji}

\usepackage{xcolor}
\newcommand{\my}[1]{
    %\textcolor{gray!80!white}{#1}
    #1
}

\usepackage{graphicx}
\graphicspath{{images/}}

\newenvironment{page}{\clearpage\normalsize\null\vfill}{\vfill\null}
\newenvironment{textblock}{\begin{minipage}{\textwidth}}{\end{minipage}\smallskip}



\begin{document}
    \begin{tikzpicture}
        \node[text width=100mm, draw=red, text justified] {
            Независимая страна с 1960 (соглашения Цюриха--Лондона). Форма правления~--- президентская республика. Президент (исполнительная власть) избирается народом каждые пять лет и управляет страной вместе с кабинетом министров, которых он сам назначает.
            C~1~мая 2004 входит в ЕC. Вступил в Еврозону 1~января 2008. Является членом ООН, не является членом НАТО. Законодательная власть осуществляется Палатой представителей (Парламентом), насчитывающей 80 мест: 56 для греко-киприотов и 24 для турко-киритов. Депутаты избираются каждые пять лет прям голосования народа. С 1963 года турко-киприоты покинули Парламент, поэтому на сегодняшний день он представлен только греко-киприотами.

            \footnotesize
            \begin{tabular}{rl}
                Πρόεδρος & Νίκος Χριστοδουλίδης\\
                \my{President} & \my{Nikos Christodoulides}\\

                Υπουργός Εξωτερικών & Κωνσταντίνος Κόμπος\\
                \my{Minister of Foreign Affairs} & \my{Konstantinos Kompos} \\

                Υπουργός Οικονομικών & Μάκης Κεραυνός\\
                \my{Minister of Finance} & \my{Makis Keravnos} \\

                Υπουργός Εσωτερικών & Κωνσταντίνος Ιωάννου\\
                \my{Minister of Interior} & \my{Konstantinos Ioannou} \\

                Υπουργός Άμυνας & Βασίλης Πάλμας\\
                \my{Minister of Defence} & \my{Vasilis Palmas} \\

                Υπουργός Παιδείας, Αθλητισμού και Νεολαίας & Αθηνά Μιχαηλίδου\\
                \my{Minister of Education, Sport and Youth} & \my{Athina Michaelidou} \\

                Υπουργός Μεταφορών, Επικοινωνιών και Έργων & Αλέξης Βαφεάδης\\
                \my{Minister of Transport, Communications and Works} & \my{Alexis Vafeades} \\

                Υπουργός Ενέργειας, Εμπορίου και Βιομηχανίας & Μιχάλης Δαμιανού\\
                \my{Minister of Energy, Commerce and Industry} & \my{Michalis Damianos} \\

                \my{(*)} Υ Γεωργίας, Αγροτικής Ανάπτυξης και Περι-ς & Μαρία Παναγιώτου\\
                \my{M of Agriculture, Rural Development and Env-t} & \my{Maria Panayiotou} \\

                Υπουργός Εργασίας και Κοινωνικών Ασφαλίσεων & Μαρίνος Μουσιούττας\\
                \my{Minister of Labour and Social Insurance} & \my{Marinos Mousiouttas} \\

                \my{(*)} Υπουργός Δικαιοσύνης και Δημόσιας Τάξεως & Κώστας Φιτσίρης\\
                \my{Minister of Justice and Public Order} & \my{Costas Fitiris} \\

                Υπουργός Υγείας & Νεόφυτος Χαραλαμπίδης\\
                \my{Minister of Health} & \my{Neophytos Charalambides} \\
            \end{tabular}

            \normalsize
            \begin{tabular}{rl}
                1960--1977 & Αρχιεπίσκοπος Μακάριος Γʹ \\
                1977--1988 & Σπύρος Κυπριανού \\
                1988--1993 & Γιώργος Βασιλείου \my{(умер в январе 2026)}\\
                1993--2003 & Γλαύκος Κληρίδης \\
                2003--2008 & Τάσσος Παπαδόπουλος \\
                2008--2013 & Δημήτρης Χριστόφιας \\
                2013--2023 & Νίκος Αναστασιάδης \\
                2023--\phantom{2026} & Νίκος Χριστοδουλίδης \\
            \end{tabular}

            Остров в~восточной части Средиземного моря, расположен близко к~Европе, Азии
            и~Африке. Он~является третьим по~величине островом в~Средиземном море после Сардинии и~Сицилии. Находится он к~востоку от~островов Родос и~Крит, к~югу
            от~берегов Турции (на~расстоянии 75 километров), к~западу от~берегов Сирии
            (на~расстоянии 105 километров) и~к~северу от~берегов Египта (на~расстоянии 380 километров).
            Кипр имеет площадь 9\,251 квадратных километров, максимальную длину 240 километров и~максимальную ширину 100 километров.
            36{,}2\% территории Кипра находится под турецкой оккупацией.

            \includegraphics[width=\textwidth]{geography1}
            \includegraphics[width=\textwidth]{geography2}
            \includegraphics[width=\textwidth]{monasteries}

            \begin{tabular}{rp{82mm}}
                01.01 &  Πρωτοχρονιά \my{(Новый год)}
                \newline Οικογενειακά γεύματα, βασιλόπιτα με φλουρί, επισκέψεις σε συγγενείς\newline
                \my{Семейные обеды, василопита с монеткой, визиты к родственникам}\\

                06.01 &  Θεοφάνεια \my{(Богоявление)}
                \newline Αγιασμός των υδάτων· στις ακτές γίνεται κατάδυση για τον σταυρό\newline
                \my{Освящение воды; на побережье — ныряние за крестом}\\

                10 μέρες &  Καρναβάλι \my{(Карнавал)}
                \newline Παρελάσεις, μεταμφιέσεις, χοροί, γιορτές (ιδιαίτερα στη Λεμεσό)\newline
                \my{Парады, маскарады, танцы, празднества (особенно в Лимассоле)}\\

                Πέμπτη &  Τσικνοπέμπτη \my{(Цикнопемпти)}
                \newline Παραδοσιακό ψήσιμο κρέατος στα κάρβουνα πριν τη Σαρακοστή\newline
                \my{Традиционная жарка мяса на углях перед Великим постом}\\

                Δευτέρα &  Καθαρά Δευτέρα \my{(Чистый понедельник)}
                \newline Πέταγμα χαρταετού, νηστίσιμα φαγητά, εξόρμηση στην ύπαιθρο\newline
                \my{Запуск воздушных змеев, постная еда, выезд на природу}\\

                25.03 &  25η Μαρτίου \my{(День греческой независимости)}
                \newline Στρατιωτικές και σχολικές παρελάσεις, σημαίες της Ελλάδας και της Κύπρου\newline
                \my{Военные и школьные парады, флаги Греции и Кипра}\\

                01.04 &  1η Απριλίου \my{(Начало борьбы EOKA)}
                \newline Επίσημες τελετές, καταθέσεις στεφάνων, σχολικές εκδηλώσεις\newline
                \my{Официальные церемонии, возложение венков, школьные мероприятия}\\

                Πάσχα &  Πάσχα \my{(Пасха)}
                \newline Αναστάσιμη λειτουργία, οικογενειακό τραπέζι, βαмμένα αυγά, ψήσιμο κρέατος· αργία τη Δευτέρα\newline
                \my{Ночная служба, семейное застолье, крашеные яйца, шашлыки; выходной понедельник}\\

                01.05 &  Πρωτομαγιά \my{(День труда)}
                \newline Πικνίκ, εξορμήσεις στη φύση, μερικές φορές συγκεντρώσεις\newline
                \my{Пикники, выезды на природу, иногда митинги}\\

                Δευτέρα &  Κατακλυσμός \my{(Катаклизмос)}
                \newline Γιορτή του Αγίου Πνεύματος, πανηγύρια στις παραλιακές πόλεις, παιχνίδια με νερό\newline
                \my{Праздник Святого Духа, ярмарки в прибрежных городах, обливания водой}\\

                15.08 &  Κοίμηση της Θεοτόκου \my{(Успение Богородицы)}
                \newline Θρησκευτικές λειτουργίες, επισκέψεις στα χωριά, οικογενειακοί εορτασμοί\newline
                \my{Церковные службы, поездки в деревни, семейные праздники}\\

                01.10 &  Ημέρα Ανεξαρτησίας \my{(День независимости Кипра)}
                \newline Στρατιωτική παρέλαση στη Λευκωσία, επίσημες εκδηλώσεις\newline
                \my{Военный парад в Никосии, официальные мероприятия}\\

                28.10 &  28η Οκτωβρίου \my{(День «Охи»)}
                \newline Παρελάσεις, μνημόσυνες τελετές, πατριωτικές εκδηλώσεις\newline
                \my{Парады, памятные церемонии, патриотические события}\\

                25.12 &  Χριστούγεννα \my{(Рождество)}
                \newline Οικογενειακά δείπνα, δώρα, εκκλησιαστικές λειτουργίες\newline
                \my{Семейные ужины, подарки, церковные службы}\\
           \end{tabular}
        };
    \end{tikzpicture}
\end{document}