% !TeX spellcheck = ru_RU-Russian
% !TEX program = lualatex
% !TEX root = greek_mobile.tex

\documentclass[12pt]{article}

\usepackage[
    paperheight=156mm, paperwidth=87.75mm,
    left=3mm, right=3mm, top=8mm, bottom=8mm, %headsep=2mm,
    showframe
]{geometry}

\usepackage{booktabs}
\usepackage{multirow}

\usepackage{fontspec}
\setmainfont[Ligatures=TeX]{Gentium Plus}

\usepackage{emoji}
\setemojifont{Apple Color Emoji}

\usepackage{xcolor}
\newcommand{\my}[1]{
    %\textcolor{gray!80!white}{#1}
    #1
}

%\usepackage{sectsty}
%\sectionfont{\color{gray}}
%\subsectionfont{\color{gray}}
%\subsubsectionfont{\color{gray}}

\usepackage{hyperref}
\hypersetup{
	hidelinks,
	%allcolors=gray,
	%colorlinks,
	pdftitle={Learning Greek},
	pdfauthor={Alexander S. Kulikov},
	pdfpagemode=FullScreen,
}

\usepackage{graphicx}
\graphicspath{{images/}}

\newenvironment{page}{\clearpage\normalsize\null\vfill}{\vfill\null}
\newenvironment{textblock}{\begin{minipage}{\textwidth}}{\end{minipage}\smallskip}



\begin{document}
	\pagestyle{empty}
    \setlength{\tabcolsep}{3pt}
    \setlength{\parindent}{0pt}

    \begin{page}
        \emoji{cyprus} Γεια σας!

        The latest version of~this pdf can be~found
        \href{https://github.com/alexanderskulikov/greek}{here}
        (last updated: \today{}).
        There, you'll also find the source file that you may use
        to~adjust this pdf for your needs. The current layout
        is~optimized for mobile devices. Table of~contents items
        are clickable.

        Enjoy!
    \end{page}

    \clearpage
    \footnotesize
	\tableofcontents
	\centering

	\begin{page}
		%\section{\emoji{input-numbers} Αριθμοί (numbers)}
		%\subsection{\emoji{input-numbers} Ακέραιοι αριθμοί (integers)}
        \section{Числа}
        \subsection{1--1000}
		\small
		\begin{tabular}{rlrlrl}
			\my{0} & μηδέν\\
			\my{1} & ένα & \my{10} & δέκα & \my{100} & εκατό\\
			\my{2} & δύο & \my{20} & είκοσι & \my{200} & διακόσια\\
			\my{3} & τρία & \my{30} & τριάντα & \my{300} & τριακόσια\\
			\my{4} & τέσσερα & \my{40} & σαράντα & \my{400} & τετρακόσια\\
			\my{5} & πέντε & \my{50} & πενήντα & \my{500} & πεντακόσια\\
			\my{6} & έξι & \my{60} & εξήντα & \my{600} & εξακόσια\\
			\my{7} & επτά & \my{70} & εβδομήντα & \my{700} & επτακόσια\\
			\my{8} & οκτώ & \my{80} & ογδόντα & \my{800} & οχτακόσια\\
			\my{9} & εννέα & \my{90} & ενενήντα & \my{900} & εννιακόσια\\
			\my{10} & δέκα & \my{100} & εκατό &\my{1000} & χίλια\\
			\my{11} & έντεκα & \my{110} & εκατόν δέκα\\
			\my{12} & δώδεκα & \my{120} & εκατόν είκοσι\\
		\end{tabular}
		\vfill
		%\subsection{\emoji{sun} Ημέρες της εβδομάδας (weekdays)}
        \subsection{Дни недели}
		\begin{tabular}{rrrl}
			\my{Sunday} & \emoji{sun} & & η Κυριακή\\
			\my{Monday} & \emoji{unamused-face} & δύο & η Δευτέρα\\
			\my{Tuesday} & \emoji{neutral-face} & τρία & η Τρίτη\\
			\my{Wednesday} & \emoji{face-without-mouth} & τέσσερα & η Τετάρτη\\
			\my{Thursday} & \emoji{slightly-smiling-face} & πέντε & η Πέμπτη\\
			\my{Friday} & \emoji{partying-face} & & η Παρασκευή\\
			\my{Saturday} & \emoji{face-with-head-bandage} & & το Σάββατο\\
		\end{tabular}
	\end{page}

	\begin{page}
		%\subsection{\emoji{watch} Ώρα (time)}
        \subsection{Время}
		\begin{tabular}{rcl}
			 & & είναι \dots \\
			14:00 & \emoji{clock2} & δύο (η ώρα) ακριβώς (το απόγευμα)\\
			14:05 & & δύο και πέντε\\
			14:15 & & δύο και δεκαπέντε\\
			 & & δύο και τέταρτο\\
			14:30 & \emoji{clock230} & δύο και τριάντα\\
			 &  & δύο και μισή\\
			 &  & δυόμιση\\
			14:35 & & δύο τριάντα πέντε\\
			 & & τρεις παρά είκοσι πέντε\\
			14:40 & & τρεις παρά είκοσι\\
			14:45 & & δύο και σαράντα πέντε\\
			 & & τρεις παρά τέταρτο\\
			15:30 & \emoji{clock330} & τρεις και τριάντα\\
			 & & τρεισήμισι\\
			16:30 & \emoji{clock430} & τεσσεράμισι\\
		\end{tabular}
	\end{page}

	\begin{page}
%		\section{\emoji{man-dancing} Ρήματα (verbs)}
%		\subsection{\emoji{performing-arts} Είμαι (to be)}
        \section{Глаголы}
        \subsection{Быть}
		\begin{tabular}{rl}
			εγώ & είμαι \\
			εσύ & είσαι\\
			αυτός/ή/ό & είναι\\
			εμείς & είμαστε\\
			εσείς & είστε\\
			αυτοί/ές/ά & είναι\\
		\end{tabular}

		\vfill

        %\subsection{\emoji{stopwatch} Ενεστώτας (present tense)}
        \subsection{Настоящее}

%        \small
%        \begin{tabular}{rrrrrr}
%            & \emoji{one} & \emoji{two} & \emoji{three} & \emoji{four} & \emoji{five}\\
%            εγώ & -ω & -άω & -ώ & -ομαι & -άμαι\\
%            εσύ &  -εις & -άς & -είς & -εσαι & -άσαι\\
%            αυτός/ή/ό &  -ει & -άει& -εί & -εται & -άται\\
%            εμείς & -ουμε & -άμε & -ούμε & -όμαστε & -όμαστε\\
%            εσείς & -ετε & -άτε & -είτε & -εστε & -άστε\\
%            αυτοί/ές/ά & -ουν& -άνε & -ούν & -ονται & -ούνται\\
%        \end{tabular}

        \scriptsize
        \begin{tabular}{rrrrrr}
            %& \emoji{one} & \emoji{two} & \emoji{three} & \emoji{four} & \emoji{five}\\
            & -ω & -άω & -ώ & -ομαι & -άμαι\\
            εγώ & \my{μέν}ω & \my{μιλ}άω & \my{οδηγ}ώ & \my{έρχ}ομαι & \my{κοιμ}άμαι\\
            εσύ &  \my{μέν}εις & \my{μιλ}άς & \my{οδηγ}είς & \my{έρχ}εσαι & \my{κοιμ}άσαι\\
            αυτός/ή/ό &  \my{μέν}ει & \my{μιλ}άει& \my{οδηγ}εί & \my{έρχ}εται & \my{κοιμ}άται\\
            εμείς &  \my{μέν}ουμε & \my{μιλ}άμε & \my{οδηγ}ούμε & \my{ερχ}όμαστε & \my{κοιμ}όμαστε\\
            εσείς &  \my{μέν}ετε & \my{μιλ}άτε & \my{οδηγ}είτε & \my{έρχ}εστε & \my{κοιμ}άστε\\
            αυτοί/ές/ά & \my{μέν}ουν& \my{μιλ}άνε & \my{οδηγ}ούν & \my{έρχ}ονται & \my{κοιμ}ούνται\\
        \end{tabular}
        \vfill\null
    \end{page}

	\begin{page}
		\subsubsection{-ω, -εις, -ει, -ουμε, -ετε, -ουν}
		\begin{tabular}{rcl}
			κάνω & \emoji{mechanic} & \my{to do}\\
			μένω & \emoji{house} & \my{to live}\\
			έχω & \emoji{money-bag} & \my{to have}\\
			δουλεύω & \emoji{office-worker} & \my{to work}\\
			ξέρω & \emoji{mage} & \my{to know}\\
			θέλω & \emoji{beer-mug} & \my{to want}\\
			πηγαίνω & \emoji{person-running} & \my{to go}\\
			καταλαβαίνω & \emoji{nerd-face} & \my{to understand}\\
			πίνω & \emoji{cup-with-straw} & \my{to drink}\\
			γράφω & \emoji{writing-hand} & \my{to write}\\
			διαβάζω & \emoji{open-book} & \my{to read}\\
			μαθαίνω & \emoji{books} & \my{to learn}\\
			σπουδάζω & \emoji{student} & \my{to study}\\
		\end{tabular}
    \end{page}

	\begin{page}
		\subsubsection{-άω, -άς, -άει, -άμε, άτε, άνε}
		\begin{tabular}{rcl}
			μιλάω & \emoji{speaking-head} & \my{to speak}\\
			αγαπάω & \emoji{smiling-face-with-hearts} & \my{to love}\\
			τραγουδάω & \emoji{singer} & \my{to sing}\\
			βοηθάω & \emoji{person-tipping-hand} & \my{to help}\\
			ξεκινάω & \emoji{rocket} & \my{to start}\\
			πεινάω & \emoji{poultry-leg} & \my{to be hungry}\\
			διψάω & \emoji{face-exhaling} & \my{to be thirsty}\\
			ξυπνάω & \emoji{yawning-face} & \my{to wake up}\\
			ρωτάω & \emoji{red-question-mark} & \my{to ask}\\
			απαντάω & \emoji{nerd-face} & \my{to answer}\\
			σταματάω & \emoji{stop-button} & \my{to stop}\\
			πονάω & \emoji{face-with-head-bandage} & \my{to be hurt}\\
		\end{tabular}
		\vfill
		\subsubsection{-ώ, -είς, -εί, -ούμε, -είτε, -ούν}
		\begin{tabular}{rcl}
			οδηγώ & \emoji{car} & \my{to drive}\\
			μπορώ & \emoji{toolbox} & \my{to can}\\
			τηλεφωνώ & \emoji{phone} & \my{to call}\\
			αργώ & \emoji{alarm-clock} & \my{to be late}\\
			χρησιμοποιώ & \emoji{mobile-phone} & \my{to use}\\
			ζω & \emoji{house} & \my{to live}
		\end{tabular}
	\end{page}

	\begin{page}
	    \subsubsection{-ομαι, -εσαι, -εται, -όμαστε, -εστε, -ονται}
	    \begin{tabular}{rcl}
	        έρχομαι & \emoji{man-walking} & \my{to come}\\
	        σκέφτομαι & \emoji{thinking-face} & \my{to think}\\
	        βρίσκομαι & \emoji{round-pushpin} & \my{находиться}\\
	        χρειάζομαι & \emoji{beer} & \my{to need}\\
	        πλένομαι & \emoji{bathtub} & \my{to bathe}\\
	        ντύνονται & \emoji{dress} & \my{to dress}\\
	        κουράζομαι & \emoji{sleeping-face} & \my{to be tired}\\
	        ξεκουράζομαι & \emoji{palm-tree} & \my{to rest}\\
	        σηκώνομαι & \emoji{man-running} & \my{to get up}\\
	        ετοιμάζομαι & \emoji{singer} & \my{to get ready}\\
	        κάθομαι & \emoji{chair} & \my{to sit down}\\
	        εργάζομαι & \emoji{office-worker} & \my{to work}\\
	    \end{tabular}
	    \vfill
	    \subsubsection{-άμαι, -άσαι, -άται, -όμαστε, -άστε, -ονται}
	    \begin{tabular}{rcl}
	        κοιμάμαι & \emoji{zzz} & \my{to sleep}\\
	        φοβάμαι & \emoji{face-screaming-in-fear} & \my{to be afraid}\\
	        λυπάμαι & \emoji{pensive-face} & \my{to be sorry} \\
	        θυμάμαι & \emoji{light-bulb} & \my{to remember}\\
	    \end{tabular}
    \end{page}

    \begin{page}
        %\subsection{\emoji{rocket} Απλός Μέλλοντας (future simple)}
        \subsection{Простое будущее}
        \tiny
        \begin{tabular}{lrcll}
            \multirow{6}{*}{\emoji{repeat}} & είμαι & \emoji{bust-in-silhouette} & θα είμαι & \my{to be}\\
            &έχω & \emoji{open-hands} & θα έχω & \my{to have}\\
            &κάνω & \emoji{gear} & θα κάνω & \my{to do}\\
            &ξέρω & \emoji{brain} & θα ξέρω & \my{to know}\\
            &περιμένω & \emoji{hourglass} & θα περιμένω & \my{to wait}\\
            &πηγαίνω & \emoji{arrow-right} & θα πάω & \my{to go}\\
            [1mm]
            \multirow{10}{*}{\emoji{one}} & διαβάζω & \emoji{book} & θα διαβάσω & \my{to read}\\
            &τελειώνω & \emoji{white-check-mark} & θα τελειώσω & \my{to finish}\\
            &ανοίγω & \emoji{unlock} & θα ανοίξω & \my{to open}\\
            &προσέχω & \emoji{warning} & θα προσέξω & \my{to pay attention}\\
            &φτιάχνω & \emoji{hammer-and-wrench} & θα φτιάξω & \my{to make}\\
            &διδάσκω & \emoji{mortar-board} & θα διδάξω & \my{to teach}\\
            &λείπω & \emoji{cry} & θα λείψω & \my{to be missing}\\
            &ανάβω & \emoji{bulb} & θα ανάψω & \my{to turn on}\\
            &γράφω & \emoji{writing-hand} & θα γράψω & \my{to write}\\
            &δουλεύω & \emoji{office} & θα δουλέψω & \my{to work}\\
            [1mm]
            \multirow{10}{*}{\emoji{two}} &μιλάω & \emoji{speech-balloon} & θα μιλήσω & \my{to speak}\\
            &αγαπάω & \emoji{heart} & θα αγαπήσω & \my{to love}\\
            &ρωτάω & \emoji{question} & θα ρωτήσω & \my{to ask}\\
            &φοράω & \emoji{t-shirt} & θα φορέσω & \my{to wear}\\
            &αρρωσταίνω & \emoji{face-with-thermometer} & θα αρρωστήσω & \my{to be ill}\\
            &γελάω & \emoji{joy} & θα γελάσω & \my{to laugh}\\
            &διψάω & \emoji{cup-with-straw} & θα διψάσω & \my{to be thirsty}\\
            [1mm]
            \multirow{5}{*}{\emoji{three}} &αργώ & \emoji{hourglass} & θα αργήσω & \my{to be late}\\
            &τηλεφωνώ & \emoji{telephone} & θα τηλεφωνήσω & \my{to call}\\
            &χρησιμοποιώ & \emoji{wrench} & θα χρησιμοποιήσω & \my{to use}\\
            &παρατηρώ & \emoji{eyes} & θα παρατηρήσω & \my{to observe}\\
            &ζω & \emoji{house} & θα ζήσω & \my{to live}\\
            [1mm]
            \multirow{4}{*}{\emoji{four}} &έρχομαι & \emoji{running} & θα έρθω & \my{to come}\\
            &γίνομαι & \emoji{sparkles} & θα γίνω &  \my{to become}\\
            &κάθομαι & \emoji{chair} & θα καθίσω & \my{to seat}\\
            [1mm]
            \multirow{1}{*}{\emoji{five}} &κοιμάμαι & \emoji{zzz} & θα κοιμηθώ & \my{to sleep}\\
            [1mm]
            \multirow{10}{*}{\emoji{triangular-flag-on-post}}
            &παίρνω & \emoji{package} & θα πάρω & \my{to take}\\
            &φεύγω & \emoji{dash} & θα φύγω & \my{to leave}\\
            &βάζω & \emoji{put-litter-in-its-place} & θα βάλω & \my{to put}\\
            &βγάζω & \emoji{door} & θα βγάλω & \my{to take out}\\
%            &μαθαίνω & \emoji{books} & θα μάθω & \my{to learn}\\
%            &καταλαβαίνω & \emoji{thought-balloon} & θα καταλάβω & \my{to understand}\\
            %
            &βλέπω & \emoji{eyes} & θα δω & \my{to see}\\
            &λέω & \emoji{speech-balloon} & θα πω & \my{to say}\\
            &πίνω & \emoji{cup-with-straw} & θα πιω & \my{to drink}\\
            &βγαίνω & \emoji{door} & θα βγω & \my{to go out}\\
            &μπαίνω & \emoji{arrow-right} & θα μπω & \my{to enter}\\
            %
            &τρώω & \emoji{fork-and-knife} & θα φάω & \my{to eat}\\
            &πηγαίνω & \emoji{arrow-right} & θα πάω & \my{to go}\\
        \end{tabular}
    \end{page}

    \begin{page}
    	\subsubsection{-ω}
        \footnotesize
    	\begin{tabular}{rclll}
    		\multicolumn{5}{l}{\my{ζ, v $\to$ σ}}\\
			διαβάζω & \emoji{book} & θα διαβάσω &                 \multirow{18}{10mm}{
                \begin{tabular}{l}
                    \emoji{one}\\
                    -ω\\
                    -εις\\
                    -ει\\
                    -ουμε\\
                    -ετε\\
                    -ουν
                \end{tabular}
            } & \my{to read}\\
			τελειώνω & \emoji{white-check-mark} & θα τελειώσω && \my{to finish}\\
            %[2mm]
			%
			\multicolumn{5}{l}{\my{γ, χ, χν, σκ $\to$ ξ}}\\
			ανοίγω & \emoji{unlock} & θα ανοίξω && \my{to open}\\
			προσέχω & \emoji{warning} & θα προσέξω && \my{to pay attention}\\
			φτιάχνω & \emoji{hammer-and-wrench} & θα φτιάξω && \my{to make}\\
			διδάσκω & \emoji{mortar-board} & θα διδάξω && \my{to teach}\\
			%[2mm]
            %
			\multicolumn{5}{l}{\my{π, β, φ $\to$ ψ}}\\
			λείπω & \emoji{cry} & θα λείψω && \my{to be missing}\\
			ανάβω & \emoji{bulb} & θα ανάψω && \my{to turn on}\\
			γράφω & \emoji{writing-hand} & θα γράψω && \my{to write}\\
			%
			\multicolumn{5}{l}{\my{εύ $\to$ έψ}}\\
            %[2mm]
			δουλεύω & \emoji{office} & θα δουλέψω && \my{to work}\\
			%
			\multicolumn{5}{l}{\my{exceptions}}\\
            %[2mm]
			κοιτάζω & \emoji{eyes} & θα κοιτάξω && \my{to look}\\
			αλλάζω & \emoji{arrows-clockwise} & θα αλλάξω && \my{to change}\\
			πειράζω & \emoji{face-with-hand-over-mouth} & θα πειράξω && \my{to tease}\\
			φωνάζω & \emoji{scream} & θα φωνάξω && \my{to scream}\\
			παίζω & \emoji{game-die} & θα παίξω && \my{to play}\\
    	\end{tabular}

        \subsubsection{-ομαι, -άμαι}
        \begin{tabular}{rclll}
            έρχομαι & \emoji{running} & θα έρθω && \my{to come}\\
            γίνομαι & \emoji{sparkles} & θα γίνω & \emoji{one} & \my{to become}\\
            κάθομαι & \emoji{chair} & θα καθίσω && \my{to seat}\\
            κοιμάμαι & \emoji{zzz} & θα κοιμηθώ & \emoji{three} & \my{to sleep}\\
        \end{tabular}
    \end{page}

    \begin{page}
        \subsubsection{-άω, -ω}
        \scriptsize
        \begin{tabular}{rclll}
            \multicolumn{5}{l}{\my{άω $\to$ ήσω}}\\
            μιλάω & \emoji{speech-balloon} & θα μιλήσω &
            \multirow{34}{8mm}{
                \begin{tabular}{l}
                    \emoji{one}\\
                    -ω\\
                    -εις\\
                    -ει\\
                    -ουμε\\
                    -ετε\\
                    -ουν
                \end{tabular}
            } & \my{to speak}\\
            αγαπάω & \emoji{heart} & θα αγαπήσω && \my{to love}\\
            ρωτάω & \emoji{question} & θα ρωτήσω && \my{to ask}\\
            απαντάω & \emoji{white-check-mark} & θα απαντήσω && \my{to answer}\\
            ξεκινάω & \emoji{rocket} & θα ξεκινήσω && \my{to start}\\
            σταματάω & \emoji{stop-sign} & θα σταματήσω && \my{to stop}\\
            ξυπνάω & \emoji{alarm-clock} & θα ξυπνήσω && \my{to wake up}\\
            συναντάω & \emoji{handshake} & θα συναντήσω && \my{to meet}\\
            βοηθάω & \emoji{handshake} & θα βοηθήσω && \my{to help}\\
            πηγαίνω & \emoji{arrow-right} & θα πάω && \my{to go}\\
            \multicolumn{5}{l}{\my{ώ $\to$ ήσω}}\\
            αργώ & \emoji{hourglass} & θα αργήσω && \my{to be late}\\
            τηλεφωνώ & \emoji{telephone} & θα τηλεφωνήσω && \my{to call}\\
            χρησιμοποιώ & \emoji{wrench} & θα χρησιμοποιήσω\hspace{-2mm} && \my{to use}\\
            παρατηρώ & \emoji{eyes} & θα παρατηρήσω && \my{to observe}\\
            ζω & \emoji{house} & θα ζήσω && \my{to live}\\
            \multicolumn{5}{l}{\my{άω $\to$ έσω}}\\
            φοράω & \emoji{t-shirt} & θα φορέσω && \my{to wear}\\
            αρρωσταίνω & \emoji{face-with-thermometer} & θα αρρωστήσω && \my{to be ill}\\
            καλώ & \emoji{telephone-receiver} & θα καλέσω && \my{to call}\\
            μπορώ & \emoji{muscle} & θα μπορέσω && \my{can}\\
            \multicolumn{5}{l}{\my{άω $\to$ άσ}}\\
            γελάω & \emoji{joy} & θα γελάσω && \my{to laugh}\\
            διψάω & \emoji{cup-with-straw} & θα διψάσω && \my{to be thirsty}\\
            πεινάω & \emoji{hamburger} & θα πεινάσω && \my{to be hungry}\\
            \multicolumn{5}{l}{\my{exceptions}}\\
            περνάω & \emoji{checkered-flag} & θα περάσω && \my{to pass}\\
            ξεχνάω & \emoji{thought-balloon} & θα ξεχάσω && \my{to forget}\\
            κερνάω & \emoji{clinking-glasses} & θα κεράσω && \my{to treat}\\
            [2mm]
            βουτάω & \emoji{diving-mask} & θα βουτήξω && \my{to dive}\\
            τραβάω & \emoji{leftwards-arrow-with-hook} & θα τραβήξω && \my{to pull}\\
            κοιτάω & \emoji{eyes} & θα κοιτάξω && \my{to look}\\
            πετάω & \emoji{airplane} & θα πετάξω && \my{to fly}\\
        \end{tabular}
    \end{page}


    \begin{page}
        %\subsubsection{\emoji{triangular-flag-on-post} Ανώμαλα ρήματα
             %(irregular verbs)}
        \subsubsection{Неправильные глаголы}
        \scriptsize
        \begin{tabular}{rclll}
            %\toprule
            πλένω & \emoji{bathtub} & θα πλύνω &
                \multirow{18}{10mm}{
                    \begin{tabular}{l}
                        \emoji{one}\\
                        -ω\\
                        -εις\\
                        -ει\\
                        -ουμε\\
                        -ετε\\
                        -ουν
                    \end{tabular}
                }
                & \my{to wash}\\
            παίρνω & \emoji{package} & θα πάρω && \my{to take}\\
            φεύγω & \emoji{dash} & θα φύγω && \my{to leave}\\
            βάζω & \emoji{put-litter-in-its-place} & θα βάλω && \my{to put}\\
            βγάζω & \emoji{door} & θα βγάλω && \my{to take out}\\
            μαθαίνω & \emoji{books} & θα μάθω && \my{to learn}\\
            καταλαβαίνω & \emoji{thought-balloon} & θα καταλάβω && \my{to understand}\\
            στέλνω & \emoji{envelope} & θα στείλω && \my{to send}\\
            δίνω & \emoji{gift} & θα δώσω && \my{to give}\\
            φέρνω & \emoji{shopping-bags} & θα φέρω && \my{to bring}\\
            θέλω & \emoji{thought-balloon} & θα θελήσω && \my{to want}\\
            μένω & \emoji{house} & θα μείνω && \my{to stay}\\
            πέφτω & \emoji{leaves} & θα πέσω && \my{to fall}\\
            παραγγέλνω & \emoji{takeout-box} & θα παραγγείλω && \my{to order}\\
            έρχομαι & \emoji{arrow-right-hook} & θα έρθω && \my{to come}\\
            γίνομαι & \emoji{seedling} & θα γίνω && \my{to become}\\
            κάθομαι & \emoji{chair} & θα καθίσω && \my{to seat}\\
            [2mm]
            %
            βλέπω & \emoji{eyes} & θα δω &
                \multirow{8}{10mm}{
                    \begin{tabular}{l}
                        \emoji{three}\\
                        -ώ\\
                        -είς\\
                        -εί\\
                        -ούμε\\
                        -είτε\\
                        -ούν
                    \end{tabular}
                }
                & \my{to see}\\
            λέω & \emoji{speech-balloon} & θα πω && \my{to say}\\
            πίνω & \emoji{cup-with-straw} & θα πιω && \my{to drink}\\
            βγαίνω & \emoji{door} & θα βγω && \my{to go out}\\
            μπαίνω & \emoji{arrow-right} & θα μπω && \my{to enter}\\
            βρίσκω & \emoji{mag} & θα βρω && \my{to find}\\
            ανεβαίνω & \emoji{arrow-up} & θα ανεβώ && \my{to go up}\\
            κατεβαίνω & \emoji{arrow-down} & θα κατεβώ && \my{to go down}\\
            [2mm]
            τρώω & \emoji{fork-and-knife} & θα φάω &
            \multirow{2}{10mm}{
                \begin{tabular}{l}
                    \emoji{two}
                \end{tabular}
            }
            & \my{to eat}\\
            πηγαίνω & \emoji{arrow-right} & θα πάω && \my{to go}\\
        \end{tabular}
    \end{page}

    \begin{page}
%        \subsection{\emoji{fast-reverse-button} Αόριστος (past tense)}
        \subsection{Прошедшее}
        \begin{tabular}{rrrrrr}
            & & κάνω & τρώω\\
            εγώ & -α & \my{έκαν}α & \my{έφαγ}α\\
            εσύ &  -ες & \my{έκαν}ες & \my{έφαγ}ες\\
            αυτός/ή/ό &  -ε & \my{έκαν}ε & \my{έφαγ}ε\\
            εμείς & -αμε & \my{κάν}αμε & \my{έφαγ}αμε\\
            εσείς & -ατε & \my{κάν}ατε & \my{έφαγ}ατε\\
            αυτοί/ές/ά & -ανε& \my{κάν}ανε & \my{έφαγ}ανε\\
            & -αν & \my{έκαν}αν & \my{έφαγ}αν\\
        \end{tabular}

        \subsubsection{-ω}
        \begin{tabular}{lrrr}
            \my{to read}  & διαβάζω & θα διαβάσω & διάβασα \\
            \my{to pay}  & πληρώνω & θα πληρώσω & πλήρωσα \\[2mm]
            \my{to open}  & ανοίγω & θα ανοίξω & άνοιξα \\
            \my{to welcome}  & προσέχω & θα προσέξω & πρόσεξα \\
            \my{to make}  & φτιάχνω & θα φτιάξω & έφτιαξα \\
            \my{to teach}  & διδάσκω & θα διδάξω & δίδαξα \\
            \my{to change}  & αλλάζω & θα αλλάξω & άλλαξα \\[2mm]
            \my{to miss}  & λείπω & θα λείψω & έλειψα \\
            \my{to write}  & γράφω & θα γράψω & έγραψα \\
            \my{to light}  & ανάβω & θα ανάψω & άναψα \\
            \my{to work} & δουλεύω & θα δουλέψω & δούλεψα \\
        \end{tabular}
    \end{page}

    \begin{page}
        \subsubsection{-άω, -ω}
        \scriptsize
        \begin{tabular}{lrrr}
            \my{to love} & αγαπάω & αγάπησα & θα αγαπήσω \\
            \my{to be late} & αργώ & άργησα & θα αργήσω \\
            \my{to help} & βοηθάω & βοήθησα & θα βοηθήσω \\
            \my{to explain} & εξηγώ & εξήγησα & θα εξηγήσω \\
            \my{to live} & ζω & έζησα & θα ζήσω \\
            \my{to request} & ζητάω & ζήτησα & θα ζητήσω \\
            \my{to speak} & μιλάω & μίλησα & θα μιλήσω \\
            \my{to wake up} & ξυπνάω & ξύπνησα & θα ξυπνήσω \\
            \my{to walk} & περπατάω & περπάτησα & θα περπατήσω \\
            \my{to ask} & ρωτάω & ρώτησα & θα ρωτήσω \\
            \my{to stop} & σταματάω & σταμάτησα & θα σταματήσω \\
            \my{to call} & τηλεφωνώ & τηλεφώνησα & θα τηλεφωνήσω \\
            \my{to sing} & τραγουδάω & τραγούδησα & θα τραγουδήσω \\
            \my{to greet} & χαιρετάω & χαιρέτησα & θα χαιρετήσω \\
            \my{to use} & χρησιμοποιώ & χρησιμοποίησα & θα χρησιμοποιήσω \\[2mm]
            \my{to laugh} & γελάω & γέλασα & θα γελάσω \\
            \my{to be thirsty} & διψάω & δίψασα & θα διψάσω \\
            \my{to be hungry} & πεινάω & πείνασα & θα πεινάσω \\
            \my{to break} & χαλάω & χάλασα & θα χαλάσω \\
            \my{to pass} & περνάω & πέρασα & θα περάσω \\
            \my{to forget} & ξεχνάω & ξέχασα & θα ξεχάσω \\[2mm]
            \my{to invite} & καλώ & κάλεσα & θα καλέσω \\
            \my{to be able} & μπορώ & μπόρεσα & θα μπορέσω \\
            \my{to hurt} & πονάω & πόνεσα & θα πονέσω \\
            \my{to wear} & φοράω & φόρεσα & θα φορέσω \\
        \end{tabular}
    \end{page}

    \begin{page}
%        \subsubsection{\emoji{triangular-flag-on-post} Ανώμαλα ρήματα
%           (irregular verbs)}
        \subsubsection{Неправильные глаголы}
        \scriptsize
        \begin{tabular}{llll}
                  & Ενεστώτας & Απλός Μέλλοντας & Αόριστος\\
            \my{go up} & ανεβαίνω & θα ανεβώ / θα ανέβω & ανέβηκα \\
            \my{put} & βάζω & θα βάλω & έβαλα \\
            \my{take out} & βγάζω & θα βγάλω & έβγαλα \\
            \my{go out} & βγαίνω & θα βγω & βγήκα \\
            \my{see} & βλέπω & θα δω & είδα \\
            \my{find} & βρίσκω & θα βρω & βρήκα \\
            \my{become} & γίνομαι & θα γίνω & έγινα \\
            \my{give} & δίνω & θα δώσω & έδωσα \\
            \my{come} & έρχομαι & θα έρθω & ήρθα \\
            \my{be} & είμαι & θα είμαι & ήμουν \\
            \my{have} & έχω & θα έχω & είχα \\
            \my{want} & θέλω & θα θέλω & ήθελα \\
            \my{do} & κάνω & θα κάνω & έκανα \\
            \my{go down} & κατεβαίνω & θα κατέβω / θα κατεβώ & κατέβηκα \\
            \my{understand} & καταλαβαίνω & θα καταλάβω & κατάλαβα \\
            \my{sit} & κάθομαι & θα καθίσω / θα κάτσω & κάθισα / έκατσα \\
            \my{sleep} & κοιμάμαι & θα κοιμηθώ & κοιμήθηκα \\
            \my{say} & λέω & θα πω & είπα \\
            \my{learn} & μαθαίνω & θα μάθω & έμαθα \\
            \my{stay} & μένω & θα μείνω & έμεινα \\
            \my{know} & ξέρω & θα ξέρω & ήξερα \\
            \my{enter} & μπαίνω & θα μπω & μπήκα \\
            \my{take} & παίρνω & θα πάρω & πήρα \\
            \my{order} & παραγγέλνω & θα παραγγείλω & παράγγειλα \\
            \my{fall} & πέφτω & θα πέσω & έπεσα \\
            \my{wait} & περιμένω & θα περιμένω & περίμενα \\
            \my{go} & πηγαίνω / πάω & θα πάω & πήγα \\
            \my{drink} & πίνω & θα πιω & ήπια \\
            \my{wash} & πλένω & θα πλύνω & έπλυνα \\
            \my{send} & στέλνω & θα στείλω & έστειλα \\
            \my{eat} & τρώω & θα φάω & έφαγα \\
            \my{leave} & φεύγω & θα φύγω & έφυγα \\
            \my{bring} & φέρνω & θα φέρω & έφερα \\
        \end{tabular}
    \end{page}

    \begin{page}
        %\subsection{\emoji{thinking-face} Απλή Υποτακτική (simple subjunctive)}
        \subsection{Простое сослагательное}
        %\small
        \begin{tabular}{lrcl}
            \my{I would like to} & θέλω / θα ήθελα &
                \multirow{10}{4mm}{
                    \begin{tabular}{c}
                        να
                    \end{tabular}
                } &
                \multirow{10}{*}{
                    \begin{tabular}{l}
                        αγοράσω\\
                        μιλήσω\\
                        δουλέψω\\
                        έρθω\\
                        φύγω\\
                        βγω\\
                    \end{tabular}
                }\\
            \my{I try to} & προσπαθώ\\
            \my{I intend to} & λέω\\
            \my{I'm about to} & πάω\\
            \my{I prefer to} & προτιμώ\\
            \my{I have to} & έχω\\
            \my{I can} & μπορώ\\
            \my{I'm thinking of} & σκέφτομαι\\
            \my{I may} & μπορεί\\
            \my{I must} & πρέπει\\
        \end{tabular}
    \end{page}

    \begin{page}
        \section{Существительные}
        %\section{\emoji{people-holding-hands} Plural}
        \subsection{Множественное}
        \begin{tabular}{rrrrrr}
            o \emoji{male-sign} & οι \emoji{male-sign}\emoji{male-sign} &
            \quad η \emoji{female-sign} & οι \emoji{female-sign}\emoji{female-sign} &
            \quad το \emoji{package} & τα \emoji{package}\emoji{package}\\
            -ης & -ες & -η & -ες & -ι & -ια\\
            -ας & -ες & -α & -ες & -o & -α\\
            -ος & -οι & -ος & -οι & -ος & -η\\
            & &&& -μα & -ματα\\
        \end{tabular}
    \end{page}

	\begin{page}
        %\section{\emoji{artist} People}
        %\section{Люди}
        %\subsection{\emoji{handshake} Meeting}
        \subsection{Знакомство}
        \scriptsize
    	\begin{tabular}{rl}%{p{32mm}p{60mm}}
    		Γεια σας! & \my{Здравствуйте!}\\
    		Γεια σου! & \my{Привет!}\\
    		Καλημέρα! & \my{Доброе утро (день)!}\\
            [2mm]
    		Πώς σε λένε; & \my{Как тебя зовут?}\\
    		Με λένε Μαρία. & \my{Меня зовут Мария.}\\
    		Εσένα; & \my{А тебя?}\\
    		Χαίρο πολύ! & \my{Приятно познакомиться!}\\
    		Χάρικα πολύ! & \my{Приятно было познакомиться!}\\
    		Επίσης! & \my{Мне тоже!}\\
    		Κι εγώ! & \my{И мне!}\\
            [2mm]
    		Από πού είσαι; & \my{Откуда ты?}\\
    		Από πού είστε; & \my{Откуда Вы?}\\
    		Είμαι από τη Ρωσία. & \my{Я из России.}\\
    		Εσύ; & \my{А ты?}\\
    		Εσείς; & \my{А Вы?}\\
    		Πού μένεις; & \my{Где живёшь?}\\
    		Μένω στην Κύπρο. & \my{Живу на Кипре.}\\
            [2mm]
    		Τι κάνεις; & \my{Как у тебя дела?}\\
    		Τι κάνετε; & \my{Как у Вас дела?}\\
    		Πώς είσαι; & \my{Как ты?}\\
    		Πώς είστε; & \my{Как Вы?}\\
    		Καλά! & \my{Хорошо!}\\
    		Πολύ καλά! & \my{Очень хорошо!}\\
    		Τέλεια! & \my{Отлично!}\\
    		Έτσι κι έτσι. & \my{Так себе.}\\
            [2mm]
    		Τι δουλειά κάνεις; & \my{Кем работаешь?}\\
    		Με τι ασχολείσαι; & \my{Кем работаешь?}\\
            [2mm]
       		Καλό μεσημέρι! & \my{Хорошего дня!}\\
    		Καλό βράδυ! & \my{Хорошего вечера!}\\
    		Καληνύχτα! & \my{Спокойной ночи!}\\
    		Τα λέμε! & \my{До встречи!}
    	\end{tabular}
	\end{page}

	\begin{page}
%		\subsection{\emoji{firefighter} Professions}
        \subsection{Профессии}
		\small
		\begin{tabular}{rrcl}
			\multicolumn{2}{r}{\emoji{male-sign}/\emoji{female-sign}}\\
			\multicolumn{2}{r}{γιατρός} & \emoji{health-worker} & \my{doctor}\\
			\multicolumn{2}{r}{μηχανικός} & \emoji{mechanic} & \my{engineer}\\
			\multicolumn{2}{r}{αστυνομικός}  & \emoji{police-officer} & \my{police-officer}\\
			\multicolumn{2}{r}{φαρμακοποιός}  & \emoji{man-health-worker} & \my{pharmacist}\\
			\multicolumn{2}{r}{τεχνικός}  & \emoji{construction-worker} & \my{technician} \\
			\multicolumn{2}{r}{δικηγόρος}  & \emoji{judge} & \my{lawyer}\\
			\multicolumn{2}{r}{υπάλληλος}  & \emoji{office-worker} & \my{employee}\\
			\multicolumn{2}{r}{διπλωμάτης}  & \emoji{briefcase} & \my{diplomat}\\
			\multicolumn{2}{r}{ταμίας}  & \emoji{euro-banknote} & \my{cashier}\\
			\multicolumn{2}{r}{γραμματέας}  & \emoji{hot-beverage} & \my{secretary}\\
			\multicolumn{2}{r}{συνταξιούχος} & \emoji{older-person} & \my{retiree}\\[2mm]
			\emoji{male-sign} & \emoji{female-sign}\\
			άνεργος & άνεργη & \emoji{star-struck} & \my{unemployed}\\
			κομμωτής & κομμώτρια & \emoji{person-getting-haircut} & \my{barber}\\
			καθηγητής & καθηγήτρια & \emoji{teacher} & \my{lecturer}\\
			πωλητής & πωλήτρια & \emoji{office-worker} & \my{salesperson}\\
			νοσηλευτής & νοσηλεύτρια & \emoji{health-worker} & \my{nurse}\\
			μαθητής & μαθήτρια & \emoji{child} & \my{learner}\\
			φοιτητής & φοιτήτρια & \emoji{student} & \my{student}\\
			δάσκαλος & δασκάλα & \emoji{teacher} & \my{teacher}\\
			ταξιτζής & ταξιτζού & \emoji{pilot} & \my{taxi driver}\\
		\end{tabular}
	\end{page}

	\begin{page}
		\vspace{-3mm}
		%\section{\emoji{cool-button} Adjectives}
        \section{Прилагательные}
		\scriptsize
		\begin{tabular}{rcl}
			καλός & \emoji{superhero} & \my{good}\\
			κακός & \emoji{supervillain} & \my{bad}\\[1mm]
			μεγάλος & \emoji{office-building} & \my{large (size), old (age)}\\
			μικρός & \emoji{microbe} & \my{small}\\
			εύκολος & \emoji{balloon} & \my{light, easy}\\
			δύσκολος & \emoji{person-lifting-weights} & \my{heavy, difficult}\\[1mm]
			όμορφος & \emoji{princess} & \my{beautiful}\\
			ωραίος & \emoji{cool-button} & \my{nice}\\
			άσχημος & \emoji{goblin} & \my{ugly}\\[2mm]
			καινούριος & \emoji{new} & \my{new}\\
			νέος & \emoji{baby} & \my{young, new}\\
			μοντέρνος & \emoji{cityscape} & \my{modern}\\
			φρέσκος & \emoji{leafy-green} & \my{fresh}\\
			παλιός & \emoji{amphora} & \my{old (things only)}\\[1mm]
			γλυκός & \emoji{lollipop} & \my{sweet}\\[1mm]
			ακριβός & \emoji{money-bag} & \my{expensive}\\
			φτηνός & \emoji{label} & \my{cheap}\\
			πλούσιος & \emoji{gem-stone} & \my{rich}\\
			φτωχός & \emoji{coin} & \my{poor}\\[2mm]
			ψηλός & \emoji{tokyo-tower} & \my{tall}\\
			κοντός & \emoji{mouse} & \my{short}\\
			λεπτός & \emoji{feather} & \my{thin}\\
			χοντρός & \emoji{wood} & \my{thick}\\[1mm]
			ζεστός & \emoji{sun} & \my{warm}\\
			κρύος & \emoji{cold-face} & \my{cold}\\[1mm]
			καθαρός & \emoji{soap} & \my{clean}\\
			βρώμικος & \emoji{litter-in-bin-sign} & \my{dirty}\\
		\end{tabular}
	\end{page}



    \begin{page}
	    %\section{\emoji{backhand-index-pointing-right} Accusative case}
        \section{Падежи}
        \begin{textblock}
            \tiny
            %\color{gray}
            \emph{Винительный:} кого? что? Часто идёт после глагола и~с~предлогами
            σε (в, на, к),
            με (с, при помощи),
            για (для, про, ради),
            μετά (после).\\
            \emph{Родительный:} кого? чего? чей? Часто идёт после существительного.
        \end{textblock}
        \smallskip

        \small
        \begin{tabular}{lrrrrrr}
            & \emoji{male-sign} & \emoji{female-sign} & \emoji{package} &
            \emoji{male-sign}\emoji{male-sign} & \emoji{female-sign}\emoji{female-sign} & \emoji{package}\emoji{package}\\
            им. & ο & η & το & οι & οι & τα\\
            вин. & τον & την & το & τους & τις & τα\\
            род. & του & της & του & των & των & των\\
        \end{tabular}
        \vfill
        \footnotesize
        \begin{tabular}{lllllllllll}
            \multirow{3}{8mm}{\emoji{male-sign}} & им. & o & πελάτης & γείτονας & φίλος\\
            & вин. & τον & πελάτη & γείτονα & φίλο\\
            & род. & του & πελάτη & γείτονα & φίλου
            \\[1mm]
            \multirow{3}{8mm}{\emoji{male-sign}\emoji{male-sign}} & им. & οι & πελάτες & γείτονες & φίλοι\\
            & вин. & τους & πελάτες & γείτονες & φίλους\\
            & род. & των & πελατών & γειτόνων & φίλων
            \\[2mm]
            \multirow{3}{8mm}{\emoji{female-sign}} & им. & η & αδελφή & γυναίκα & άσκηση\\
            & вин. & την & αδελφή & γυναίκα & άσκηση\\
            & род. & της & αδελφής & γυναίκας & άσκησης
            \\[1mm]
            \multirow{3}{8mm}{\emoji{female-sign}\emoji{female-sign}} & им. & οι & αδελφές & γυναίκες & ασκήσεις\\
            & вин. & τις & αδελφές & γυναίκες & ασκήσεις\\
            & род. & των & αδελφών & γυναικών & ασκήσεων
            \\[2mm]
            \multirow{3}{8mm}{\emoji{package}} & им. & το & δώρο & παιδί & πρόβλημα\\
            & вин. & το & δώρο & παιδί & πρόβλημα\\
            & род. & του & δώρου & παιδιού & προβλήματος
            \\[1mm]
            \multirow{3}{8mm}{\emoji{package}\emoji{package}} & им. & τα & δώρα & παιδιά & προβλήματα\\
            & вин. & τα & δώρα & παιδιά & προβλήματα\\
            & род. & των & δώρων & παιδιών & προβλημάτων
        \end{tabular}
        \vspace{-5mm}
    \end{page}

    \begin{page}
        \subsection{Винительный}
        \begin{textblock}
        	\tiny
        	Кого? Что? Часто идёт после глагола и~с~такими предлгами:
            σε (в, на, к),
            με (с, при помощи),
            για (для, про, ради),
            μετά (после).
        \end{textblock}
	    \footnotesize
	    \begin{tabular}{rrrr}
	        \emoji{male-sign} &
	        \emoji{male-sign}\emoji{male-sign} &
	        \emoji{backhand-index-pointing-right}\emoji{male-sign} & \emoji{backhand-index-pointing-right}\emoji{male-sign}\emoji{male-sign}\\
	        ο \my{αθλητ}ής & οι \my{αθλητ}ές & τον \my{αθλητ}ή & τους \my{αθλητ}ές\\
	        ο \my{άνδρ}ας & οι \my{άνδρ}ες & τον \my{άνδρ}α & τους \my{άνδρ}ες\\
	        ο \my{φίλ}ος & οι \my{φίλ}οι & τον \my{φίλ}ο & τους \my{φίλ}ους\\
	        \my{αυτ}ός & \my{αυτ}οί & \my{αυτ}όν & \my{αυτ}ούς\\
	        \my{εκείν}ος & \my{εκείν}οι & \my{εκείν}ον & \my{εκείν}ους\\
	        \my{πόσ}ος & \my{πόσ}οι & \my{πόσ}ο(ν) & \my{πόσ}ους\\
	        \my{πολ}ύς & \my{πολ}λοί & \my{πολ}ύ & \my{πολ}λούς\\
	        [2mm]
	        %
	        \emoji{female-sign} &
	        \emoji{female-sign}\emoji{female-sign} &
	        \emoji{backhand-index-pointing-right}\emoji{female-sign} & \emoji{backhand-index-pointing-right}\emoji{female-sign}\emoji{female-sign}\\
	        η \my{φίλ}η & οι \my{φίλ}ες & τη \my{φίλ}η & τις \my{φίλ}ες\\
	        η \my{γυναίκ}α & οι \my{γυναίκ}ες & τη \my{γυναίκ}α & τις \my{γυναίκ}ες\\
	        η \my{οδ}ός & οι \my{οδ}οί & την \my{οδ}ό & τις \my{οδ}ούς\\
	        \my{αυτ}ή & \my{αυτ}ές & \my{αυτ}ή & \my{αυτ}ές\\
	        \my{εκείν}η & \my{εκείν}ες & \my{εκείν}η & \my{εκείν}ες\\
	        \my{πόσ}η & \my{πόσ}ες & \my{πόσ}η & \my{πόσ}ες\\
	        \my{πολ}λή & \my{πολ}λές & \my{πολ}λή & \my{πολ}λές\\
	        [2mm]
	        %
	        \emoji{package} &
	        \emoji{package}\emoji{package} &
	        \emoji{backhand-index-pointing-right}\emoji{package} & \emoji{backhand-index-pointing-right}\emoji{package}\emoji{package}\\
	        το \my{δώρ}ο & τα \my{δώρ}α & το \my{δώρ}ο & τα \my{δώρ}α\\
	        το \my{παιδ}ί & τα \my{παιδ}ιά & το \my{παιδ}ί & τα \my{παιδ}ιά\\
	        το \my{μάθη}μα & τα \my{μαθή}ματα & το \my{μάθη}μα & τα \my{μαθή}ματα\\
	        το \my{λάθ}ος & τα \my{λάθ}η & το \my{λάθ}ος & τα \my{λάθ}η\\
	        \my{αυτ}ό & \my{αυτ}ά & \my{αυτ}ό & \my{αυτ}ά\\
	        \my{εκείν}ο & \my{εκείν}α & \my{εκείν}ο & \my{εκείν}α\\
	        \my{πόσ}ο & \my{πόσ}α & \my{πόσ}ο & \my{πόσ}α\\
	        \my{πολ}ύ & \my{πολ}λά & \my{πολ}ύ & \my{πολ}λά
	    \end{tabular}
    \end{page}

    \begin{page}
        \subsection{Родительный}

        \begin{textblock}
            \tiny
            Кого? Чего? Чей? Почти всегда идёт после существительного.\\
            Ο Γιάννης έχει ένα σπίτι. Το σπίτι του είναι μικρό. Το σπίτι του Γιάννη είναι μικρό.\\
            Η Ελένη έχει ένα σπίτι. Το σπίτι της είναι μικρό. Το σπίτι της Ελένης είναι μικρό.\\
            Το παιδί έχει ένα σπίτι. Το σπίτι του είναι μικρό. Το σπίτι του παιδιού είναι μικρό.
        \end{textblock}


        \vspace{5mm}

        \begin{tabular}{rrr}
            του \emoji{male-sign} & της \emoji{female-sign} & του \emoji{package}\\
            \my{Κώστ}α & \my{Άνν}ας & \my{παιδ}ιού\\
            \my{πίνακ}α & \my{εικόν}ας & \my{μολυβ}ιού\\
            \my{Γιάνν}η & \my{Ελέν}ης & \my{μωρ}ού\\
            \my{χάρτ}η & \my{ζών}ης & \my{βιβλί}ου\\
            \my{Πέτρ}ου & \my{τάξ}ης & \my{μαθή}ματος\\
            \my{δρόμ}ου\\[3mm]
            των \emoji{male-sign}\emoji{male-sign} & των \emoji{female-sign}\emoji{female-sign} & των \emoji{package}\emoji{package}\\
            \my{χειμών}ων & \my{θαλασσ}ών & \my{παιδ}ιών\\
            \my{πιν}ά\my{κ}ων & \my{εικόν}ων & \my{τραγουδ}ιών\\
            \my{χαρτ}ών & \my{ζων}ών & \my{μωρ}ών\\
            \my{τοίχ}ων & & \my{βιβλί}ων\\
            \my{δασκάλ}ων && \my{θε}ά\my{τρ}ων\\
            && \my{μαθη}μάτων
        \end{tabular}
    \end{page}

    \begin{page}
        %\section{\emoji{atom-symbol} Conjunctions and particles}
        \section{Разное}
        \begin{tabular}{rl}
            Όμως & \my{but, however}\\
            Όταν & \my{when}\\
            Αν & \my{if}\\
            Και & \my{and}\\
            Γιατί & \my{because, why}\\
            Ή & \my{or}\\
            Που & \my{that, which, who}\\
            Μετά & \my{after}\\
            Πριν & \my{before}\\
            Επειδή & \my{because}\\
            Μόλις & \my{as soon as}\\
            Ενώ & \my{while, whereas}\\
            Αφού & \my{since, after}\\
            Όπου & \my{where}\\
            Ωστόσο & \my{nevertheless, yet}\\
        \end{tabular}
    \end{page}

    \begin{page}
        %\section{\emoji{owl} Τι; Πού; Πότε; (What? Where? When?)}
        \subsection{Что? Где? Когда?}
        \scriptsize
        \setlength{\tabcolsep}{2pt}
        \null\hspace{-2mm}
        \begin{tabular}{rllllll}
            & \emoji{male-sign} & \emoji{female-sign} & \emoji{package} &
            \emoji{male-sign}\emoji{male-sign} & \emoji{female-sign}\emoji{female-sign} & \emoji{package}\emoji{package}\\[2mm]
            \multicolumn{7}{c}{именительный}\\[1mm]
            \my{Who} & Ποιος & Ποια & Ποιο &  Ποιοι & Ποιες & Ποια\\
            \my{What} & Τι & Τι & Τι & Τι & Τι & Τι\\
            \my{Where} & Πού & Πού & Πού & Πού & Πού & Πού\\
            \my{When} & Πότε & Πότε & Πότε & Πότε & Πότε & Πότε\\
            \my{Why} & Γιατί & Γιατί & Γιατί & Γιατί & Γιατί & Γιατί\\
            \my{What for} & Για ποιον & Για ποια & Για ποιο & Για ποιους & Για ποιες & Για ποια\\
            \my{Which} & Ποιος & Ποια & Ποιο & Ποιοι & Ποιες & Ποια\\
            \my{Whose} & Ποιανού & Ποιανής & Ποιανού & Ποιανών & Ποιανών & Ποιανών\\[2mm]
			%
			\multicolumn{7}{c}{винительный}\\[1mm]
            \my{Whom} & Ποιον & Ποια & Ποιο & Ποιους & Ποιες & Ποια\\
            \my{What} & Τι & Τι & Τι & Τι & Τι & Τι\\
            \my{Where} & Πού & Πού & Πού & Πού & Πού & Πού\\
            \my{When} & Πότε & Πότε & Πότε & Πότε & Πότε & Πότε\\
            \my{Why} & Γιατί & Γιατί & Γιατί & Γιατί & Γιατί & Γιατί\\
            \my{What for} & Για ποιον & Για ποια & Για ποιο & Για ποιους & Για ποιες & Για ποια\\
            \my{Which} & Ποιον & Ποια & Ποιο & Ποιους & Ποιες & Ποια\\[2mm]
            %
            \multicolumn{7}{c}{родительный}\\[1mm]
        	\my{Who} & Ποιου & Ποιας & Ποιου & Ποιων & Ποιων & Ποιων\\
        	\my{What} & Τινος & Τινος & Τινος & Τινος & Τινος & Τινος\\
        	\my{Where} & Πού & Πού & Πού & Πού & Πού & Πού\\
        	\my{When} & Πότε & Πότε & Πότε & Πότε & Πότε & Πότε\\
        	\my{Why} & Γιατί & Γιατί & Γιατί & Γιατί & Γιατί & Γιατί\\
        	\my{What for} & Για ποιου & Για ποιας & Για ποιου & Για ποιων & Για ποιων & Για ποιων\\
        	\my{Which} & Ποιου & Ποιας & Ποιου & Ποιων & Ποιων & Ποιων\\
        	\my{Whose} & Ποιανού & Ποιανής & Ποιανού & Ποιανών & Ποιανών & Ποιανών\\
        \end{tabular}
    \end{page}

    \begin{page}
        %\section{\emoji{round-pushpin} Προθέσεις (prepositions)}
        \subsection{Предлоги}
        \begin{tabular}{rl}
            από & \my{from, since, by}\\
            σε & \my{in, at, to}\\
            με & \my{with, by}\\
            για & \my{for, about}\\
            χωρίς & \my{without}\\
            ως & \my{as, up to}\\
            μέχρι & \my{until, up to}\\
        \end{tabular}
        \vfill
        \scriptsize
        \begin{tabular}{rl}
            Ήρθα από την Αθήνα & \my{I came from Athens}\\
            Είναι σε ένα καφέ & \my{He is at a café}\\
            Ταξιδεύω με το τρένο & \my{I travel by train}\\
            Αυτό το δώρο είναι για σένα & \my{This gift is for you}\\
            Πίνω καφέ χωρίς ζάχαρη & \my{I drink coffee without sugar}\\
            Δούλευε ως δάσκαλος & \my{He worked as a teacher}\\
            Θα μείνω εδώ μέχρι το βράδυ & \my{I will stay here until the evening}
        \end{tabular}
    \end{page}

    \begin{page}
    	\null\vspace{-30mm}
    	\section{Культура}
    	\subsection{Политика}
    	\tiny

    	\begin{textblock}
    		Независимая страна с 1960 (соглашения Цюриха--Лондона). Форма правления~--- президентская республика. Президент (исполнительная власть) избирается народом каждые пять лет и управляет страной вместе с кабинетом министров, которых он сам назначает.
    		C~1~мая 2004 входит в ЕC. Вступил в Еврозону 1~января 2008. Является членом ООН, не является членом НАТО. Законодательная власть осуществляется Палатой представителей (Парламентом), насчитывающей 80 мест: 56 для греко-киприотов и 24 для турко-киритов. Депутаты избираются каждые пять лет прям голосования народа. С 1963 года турко-киприоты покинули Парламент, поэтому на сегодняшний день он представлен только греко-киприотами.
    	\end{textblock}
        %\vspace{-5mm}
        %\subsubsection{Совет министров}
        \begin{tabular}{rl}
        	Πρόεδρος & Νίκος Χριστοδουλίδης\\
        	\my{President} & \my{Nikos Christodoulides}\\

        	Υπουργός Εξωτερικών & Κωνσταντίνος Κόμπος\\
        	\my{Minister of Foreign Affairs} & \my{Konstantinos Kompos} \\

        	Υπουργός Οικονομικών & Μάκης Κεραυνός\\
        	\my{Minister of Finance} & \my{Makis Keravnos} \\

        	Υπουργός Εσωτερικών & Κωνσταντίνος Ιωάννου\\
        	\my{Minister of Interior} & \my{Konstantinos Ioannou} \\

        	Υπουργός Άμυνας & Βασίλης Πάλμας\\
        	\my{Minister of Defence} & \my{Vasilis Palmas} \\

        	Υπουργός Παιδείας, Αθλητισμού και Νεολαίας & Αθηνά Μιχαηλίδου\\
        	\my{Minister of Education, Sport and Youth} & \my{Athina Michaelidou} \\

        	Υπουργός Μεταφορών, Επικοινωνιών και Έργων & Αλέξης Βαφεάδης\\
        	\my{Minister of Transport, Communications and Works} & \my{Alexis Vafeades} \\

        	Υπουργός Ενέργειας, Εμπορίου και Βιομηχανίας & Μιχάλης Δαμιανού\\
        	\my{Minister of Energy, Commerce and Industry} & \my{Michalis Damianos} \\

        	\my{(*)} Υπουργός Γεωργίας, Αγροτικής Ανάπτυξης και Περιβάλλοντος & Μαρία Παναγιώτου\\
        	\my{Minister of Agriculture, Rural Development and Environment} & \my{Maria Panayiotou} \\

        	Υπουργός Εργασίας και Κοινωνικών Ασφαλίσεων & Μαρίνος Μουσιούττας\\
        	\my{Minister of Labour and Social Insurance} & \my{Marinos Mousiouttas} \\

        	\my{(*)} Υπουργός Δικαιοσύνης και Δημόσιας Τάξεως & Κώστας Φιτσίρης\\
        	\my{Minister of Justice and Public Order} & \my{Costas Fitiris} \\

        	Υπουργός Υγείας & Νεόφυτος Χαραλαμπίδης\\
        	\my{Minister of Health} & \my{Neophytos Charalambides} \\
        \end{tabular}
        %\null\vspace{-2mm}
        %\subsubsection{Президенты}
        \begin{tabular}{rl}
        	1960--1977 & Αρχιεπίσκοπος Μακάριος Γʹ \\
        	1977--1988 & Σπύρος Κυπριανού \\
        	1988--1993 & Γιώργος Βασιλείου \my{(умер в январе 2026)}\\
        	1993--2003 & Γλαύκος Κληρίδης \\
        	2003--2008 & Τάσσος Παπαδόπουλος \\
        	2008--2013 & Δημήτρης Χριστόφιας \\
        	2013--2023 & Νίκος Αναστασιάδης \\
        	2023--\phantom{2026} & Νίκος Χριστοδουλίδης \\
        \end{tabular}
    \end{page}

    \begin{page}
        \null\vspace{-5mm}
    	\subsection{География}
        \vspace{-3mm}
    	\begin{textblock}
    		\tiny
    		Остров в~восточной части Средиземного моря, расположен близко к~Европе, Азии
    		и~Африке. Он~является третьим по~величине островом в~Средиземном море после Сардинии и~Сицилии. Находится он к~востоку от~островов Родос и~Крит, к~югу
    		от~берегов Турции (на~расстоянии 75 километров), к~западу от~берегов Сирии
    		(на~расстоянии 105 километров) и~к~северу от~берегов Египта (на~расстоянии 380 километров).
    		Кипр имеет площадь 9\,251 квадратных километров, максимальную длину 240 километров и~максимальную ширину 100 километров.
    		36{,}2\% территории Кипра находится под турецкой оккупацией.
    	\end{textblock}
    	\includegraphics[width=.6\textwidth]{geography1}
    	\includegraphics[width=.6\textwidth]{geography2}
    	\includegraphics[width=.6\textwidth]{monasteries}
    \end{page}

	\begin{page}
		\subsection{Даты}
		\subsubsection{Праздники}
		\tiny
		\begin{tabular}{rp{68mm}}
			01.01 &  Πρωτοχρονιά \my{(Новый год)}
			\newline Οικογενειακά γεύματα, βασιλόπιτα με φλουρί, επισκέψεις σε συγγενείς\newline
			\my{Семейные обеды, василопита с монеткой, визиты к родственникам}\\

			06.01 &  Θεοφάνεια \my{(Богоявление)}
			\newline Αγιασμός των υδάτων· στις ακτές γίνεται κατάδυση για τον σταυρό\newline
			\my{Освящение воды; на побережье — ныряние за крестом}\\

			10 μέρες &  Καρναβάλι \my{(Карнавал)}
			\newline Παρελάσεις, μεταμφιέσεις, χοροί, γιορτές (ιδιαίτερα στη Λεμεσό)\newline
			\my{Парады, маскарады, танцы, празднества (особенно в Лимассоле)}\\

			Πέμπτη &  Τσικνοπέμπτη \my{(Цикнопемпти)}
			\newline Παραδοσιακό ψήσιμο κρέατος στα κάρβουνα πριν τη Σαρακοστή\newline
			\my{Традиционная жарка мяса на углях перед Великим постом}\\

			Δευτέρα &  Καθαρά Δευτέρα \my{(Чистый понедельник)}
			\newline Πέταγμα χαρταετού, νηστίσιμα φαγητά, εξόρμηση στην ύπαιθρο\newline
			\my{Запуск воздушных змеев, постная еда, выезд на природу}\\

			25.03 &  25η Μαρτίου \my{(День греческой независимости)}
			\newline Στρατιωτικές και σχολικές παρελάσεις, σημαίες της Ελλάδας και της Κύπρου\newline
			\my{Военные и школьные парады, флаги Греции и Кипра}\\

			01.04 &  1η Απριλίου \my{(Начало борьбы EOKA)}
			\newline Επίσημες τελετές, καταθέσεις στεφάνων, σχολικές εκδηλώσεις\newline
			\my{Официальные церемонии, возложение венков, школьные мероприятия}\\

			Πάσχα &  Πάσχα \my{(Пасха)}
			\newline Αναστάσιμη λειτουργία, οικογενειακό τραπέζι, βαмμένα αυγά, ψήσιμο κρέατος· αργία τη Δευτέρα\newline
			\my{Ночная служба, семейное застолье, крашеные яйца, шашлыки; выходной понедельник}\\

			01.05 &  Πρωτομαγιά \my{(День труда)}
			\newline Πικνίκ, εξορμήσεις στη φύση, μερικές φορές συγκεντρώσεις\newline
			\my{Пикники, выезды на природу, иногда митинги}\\

			Δευτέρα &  Κατακλυσμός \my{(Катаклизмос)}
			\newline Γιορτή του Αγίου Πνεύματος, πανηγύρια στις παραλιακές πόλεις, παιχνίδια με νερό\newline
			\my{Праздник Святого Духа, ярмарки в прибрежных городах, обливания водой}\\

			15.08 &  Κοίμηση της Θεοτόκου \my{(Успение Богородицы)}
			\newline Θρησκευτικές λειτουργίες, επισκέψεις στα χωριά, οικογενειακοί εορτασμοί\newline
			\my{Церковные службы, поездки в деревни, семейные праздники}\\

			01.10 &  Ημέρα Ανεξαρτησίας \my{(День независимости Кипра)}
			\newline Στρατιωτική παρέλαση στη Λευκωσία, επίσημες εκδηλώσεις\newline
			\my{Военный парад в Никосии, официальные мероприятия}\\

			28.10 &  28η Οκτωβρίου \my{(День «Охи»)}
			\newline Παρελάσεις, μνημόσυνες τελετές, πατριωτικές εκδηλώσεις\newline
			\my{Парады, памятные церемонии, патриотические события}\\

			25.12 &  Χριστούγεννα \my{(Рождество)}
			\newline Οικογενειακά δείπνα, δώρα, εκκλησιαστικές λειτουργίες\newline
			\my{Семейные ужины, подарки, церковные службы}\\
		\end{tabular}
	\end{page}

	\begin{page}
		\subsubsection{Важные даты}
		%\scriptsize
		\tiny
        \begin{tabular}{rp{64mm}}
			16.08.1960 & Ημέρα της Ανεξαρτησίας της Κύπρου\newline
			\my{День независимости Кипра}\\

			21.12.1963 & Ματωμένα Χριστούγεννα\newline
			\my{Кровавое рождество (начало межобщинных столкновений)}\\

			04.03.1964 & Άφιξη της Ειρηνευτικής Δύναμης των Ηνωμένων Εθνών\newline
			\my{Размещение миротворческих сил ООН (ВСООНК)}\\

			15.07.1974 & Το Πραξικόπημα της 15ης Ιουλίου\newline
			\my{Государственный переворот (свержение Макариоса III)}\\

			20.07.1974 & Η Τουρκική Εισβολή στην Κύπρο\newline
			\my{Начало турецкого военного вторжения (Операция Аттила)}\\

			14.08.1974 & Η δεύτερη φάση της τουρκικής εισβολής\newline
			\my{Вторая фаза вторжения и оккупация севера острова}\\

			03.08.1977 & Θάνατος του Αρχιεπισκόπου Μακαρίου Γ'\newline
			\my{Смерть первого президента Кипра, архиепископа Макариоса III}\\

			15.11.1983 & Η Μονομερής Ανακήρυξη του Ψευδοκράτους\newline
			\my{Одностороннее провозглашение «независимости» оккупированных территорий}\\

			14.08.1996 & Η δολοφονία του Σολωμού Σολωμού\newline
			\my{Убийство Соломоса Солому во время протестов в Деринии}\\

			23.04.2003 & Το πρώτο άνοιγμα των οδοφραγμάτων\newline
			\my{Первое открытие контрольно-пропускных пунктов на границе}\\

			24.04.2004 & Το Δημοψήφισμα για το Σχέδιο Ανάν\newline
			\my{Референдум по Плану Аннана по объединению острова}\\

			01.05.2004 & Η Ένταξη της Κύπρου στην Ευρωπαϊκή Ένωση\newline
			\my{Вступление Республики Кипр в Европейский Союз}\\

			01.01.2008 & Η Υιοθέτηση του Ευρώ στην Κύπρο\newline
			\my{Официальный переход Кипра на валюту евро}\\

			11.07.2011 & Η έκρηξη στο Μαρί\newline
			\my{Взрыв на военной базе «Евангелос Флоракис» в Мари}\\

			28.12.2011 & Ανακάλυψη Φυσικού Αερίου στο κοίτασμα «Αφροδίτη»\newline
			\my{Официальное объявление об открытии газового месторождения «Афродита»}\\
		\end{tabular}
	\end{page}

	\begin{page}
		\subsection{Жизнь}
		\tiny
		\begin{textblock}
			\begin{description}
				\item[Официальные языки.] Греческий и~турецкий.

                \item[Религии.]
				Греко-киприоты~--- православные христиане, турко-киприоты~--- мусульмане. Также проживают армяне, марониты (прибыли из~Ливана в~8--13 веках) и~приверженцы римско-католической церкви.

				\item[Дороги.] Макс. скорость на магистрали~--- 100 км/ч, макс. кол-во алкоголя~--- 22 мг на 100 мл выдыхаемого воздуха.

                \item[Электричество.] Интерконнектор соединяет Кипр, Грецию, Израиль. Также строят LNG terminal в Василикосе.

                \item[ВВП.] Процент ВВП от сельского хозяйства~--- 2\%.

                \item[ГеСИ.] 18--40 лет~--- 4~бесплатных посещения; 47--50 лет~--- 6~бесплатных посещений; дальше~--- 15~евро за раз. Приём у~врачей-специалистов стоит 6~евро, у~рентгенологов~--- 10.
                Одна бесплатная чистка зубов в~год.

				\item[Газеты.] Ежедневные: Филеефтерос,
				Политис, Симерини и~Харавги; по воскресеньям: Кафимерини;
                англоязычная: Сyprus Mail.

				\item[Телекоммуникации.] Код: 00357 (+357); телефонные компании: Cyta (АТЕН), Epic, Primetel и~Cablenet; гос. оператор: РИК.

                \item[Национальная Гвардия.] Объединенные вооруженные силы (сухопутные, морские и~авиационные). Продолжительность службы~--- 14~месяцев; для женщин-волонтёров~--- 6~месяцев.

%                Служат только греко-киприоты. С~2008 года военная служба обязательна для всех членов Греческого Сообщества, а не только для этнических греков.
%				Продолжительность службы составляет 14 месяцев.
				%Все мужчины в~возрасте
				%от~16~лет и~старше, являющиеся киприотами по~происхождению, обязаны получить разрешение на~выезд из~Республики Кипр от~Министерства Обороны.

				\item[Спорт.] Футбольные команды: АЕЛ (Лимассол), АПОЭЛ (Никосия), Омония (Никосия)
				и~Анортосис (Фамагуста).

				\item[Учреждения и часы работы.]
                С пн по пт, с~7:30--8:00 до до~14:00--15:00 дня.
				Аптеки: до~13:30 по ср и сб, в вск не работают, летом ночные работают до 23:00, дежурные~--- по телефону 11892.
				Банки: 8:00--14:30. Организации:
                АНК~--- Управление электроэнергетики кипра,
				Департаменты водоснабжения Кипра,
				КЕП~--- Центр обслуживания граждан (типа МФЦ).
			\end{description}
		\end{textblock}
	\end{page}
\end{document}